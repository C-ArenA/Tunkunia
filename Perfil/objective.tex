\section{Objetivos}

Desarrollar la versión inicial de un paquete de software libre de código y
concepto reutilizable, \textit{FOSS (free and open source software)}, que facilite la
tarea de creación y seguimiento de trámites en base al uso de autómatas finitos
o máquinas de estado finitas, la definición de lineamientos para su
implementación y utilidades comunes a estos procedimientos administrativos. Todo
esto en base a los requerimientos, experiencias y desarrollo del sistema
\say{SIAI} (Sistema de Información Ambiental Industrial) del \say{Ministerio de
Desarrollo Productivo y Economía Plural}, que fue implementado por la empresa
2IES utilizando el \textit{framework Laravel}.

La concreción de este objetivo involucrará de forma necesaria la realización de
las actividades siguientes:

\begin{itemize}
	\item Extraer los requerimientos específicos al módulo de trámites del
	      sistema SIAI, que es un sistema que cuenta con varios módulos y los
	      procesos administrativos son parte de uno de ellos.

	\item Definir una licencia de software libre que tenga coherencia con el
	      objetivo general del proyecto, que armonice con los preceptos defendidos
	      por la \textit{FSF (Free Software Foundation)} y que sea compatible con la
	      normativa boliviana.

	\item Estudiar y analizar casos de éxito en el mundo del \textit{software open
	      source}, considerando factores relevantes como el número de descargas,
	      \textit{forks}, etc \cite{mujahidWhatAreCharacteristics2023}.

	\item Adoptar prácticas que permitan entregar software de calidad.

	\item Escribir código autodocumentado y cuando sea necesario correctamente
	      comentado para ser amigable a las contribuciones.

	\item Recibir al menos un \textit{pull request} de un colaborador, como prueba de
	      legibilidad del código.

	\item Redacción y despliegue de una documentación clara y moderna que sea
	      fácilmente navegable y entendible, atendiendo a la reusabilidad del
	      paquete a desarrollar.

	\item Poner en práctica patrones de diseño comunes como el patrón estado
	      basado en máquinas de estados.

	\item Usar el paquete de software resultante de este proyecto en una
	      implementación de un \textit{fork} del sistema SIAI como caso de estudio.

	\item Implementar una interfaz para facilitar al desarrollador el uso de la
	      herramienta.

	\item Publicar tanto este documento, realizado en Latex, como el documento
	      final de proyecto y el código fuente en un repositorio remoto como
		  \textit{GitHub}.

	\item Añadir el paquete realizado a un repositorio de paquetes como el
	      utilizado por el \textit{framework Laravel}.

	\item Implementar medidas para garantizar la integridad de los datos de los
	      trámites dentro del sistema.

	\item Integrarse con herramientas de software libre ya existentes dentro del
	      entorno sobre el cual se realice el desarrollo para facilitar y mejorar
	      la calidad de la implementación de este proyecto.

	\item Incentivar una nueva serie de proyectos de grado sobre software
	      reutilizable \textit{open source} en la carrera de Ingeniería Electrónica para
	      posicionar a los estudiantes en el mercado laboral y como profesionales

	\item Crear una comunidad de proyectos de código abierto alrededor de éste.
\end{itemize}