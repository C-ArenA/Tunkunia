\section{Planteamiento del Problema}
Gobiernos de todo el mundo buscan digitalizar sus procesos para facilitar la interacción con la población y aumentar la eficiencia de los mismos, apuntando a lo que la CEPAL llama Gobiernos Digitales y Gobiernos Inteligentes. 

Dentro de todos los procesos administrativos la figura de trámite se encuentra siempre presente a todo nivel y suele ser motivo o parte del motivo de la creación de muchos sistemas de software gubernamentales.

El desarrollo de sistemas de software es un proceso largo que requiere por sí mismo la toma de varias decisiones profesionales para brindar a los usuarios el mejor producto en el menor tiempo posible y garantizando que se cumplan ciertos requerimientos funcionales y no funcionales, implicando una gran complejidad que sería limitante si no fuera por algunas buenas prácticas de desarrollo.

A menudo, los equipos de desarrollo, para evitar que la complejidad de los sistemas sea excesiva, buscan herramientas de terceros que les permiten no reinventar la rueda. Estas herramientas, llamadas paquetes de software, que pueden ser librerías o incluso frameworks de desarrollo, ofrecen una serie de ventajas a los proyectos de desarrollo, así como algunas limitaciones que, en la mayoría de los casos, y dependiendo del proyecto, no son relevantes, pero requieren la atención del equipo de profesionales.

Estas herramientas, que forman parte fundamental del desarrollo moderno, suelen ser de código abierto, o al menos así se prefieren por muchos profesionales, dadas sus ventajas de seguridad y versatilidad.

Es importante notar también que en algunos países se demanda o prefiere el uso de software libre en el ámbito público. Tal es el caso de Bolivia, que en su artículo 77 de la ley general de telecomunicaciones, tecnologías de información y comunicación indica que se debe promover y priorizar el uso del mismo dentro de los órganos ejecutivo, legislativo, judicial y electoral.

Las tecnologías usadas para el desarrollo de estos sistemas son abundantes, y aunque proporcionan utilidades para facilitar la creación de los mismos, no proporcionan de manera específica un marco de trabajo para trabajar con los trámites administrativos.

Además, al ser un asunto común a muchos de estos sistemas implica que cada desarrollo reinvente la rueda y tome más tiempo. Cada nuevo desarrollo puede o no contar con todas las herramientas necesarias para la creación de un sistema de trámites.

El desarrollo de un sistema puede utilizar distintos lenguajes de programación así como distintos frameworks de desarrollo. Uno de los frameworks que constantemente gana popularidad y madurez es el llamado Laravel basado en PHP, que además de ser el framework de PHP más utilizado al día de hoy, es también uno de los más prácticos, completos y de rápido desarrollo. Por lo tanto, es muy utilizado y recomendado.

Tras una búsqueda en la red no se encuentran librerías específicas al desarrollo de sistemas de trámites como módulo dentro de otro sistema, a pesar de ser una necesidad obvia y frecuente en sistemas de gobiernos.

Las soluciones actuales para realización de trámites implican desarrollos desde cero, menor eficiencia, falta de estándares claros \parentext{Como consecuencia varios sistemas fallidos} y reinvención constante de la rueda.

Existen, sin embargo, sistemas completos para este propósito, los cuales se venden o alquilan, pero carecen de versatilidad, no son un submódulo en un sistema mayor, no son personalizados y generalmente no son de código abierto, siendo este último un requisito en la realización de sistemas gubernamentales en algunos países como Bolivia.

Cuando una funcionalidad es común en distintos sistemas, una solución viable y común es la creación de una herramienta reutilizable de código abierto que ayude en su implementación. Esto proporciona varias ventajas:

\begin{itemize}
    \item Tiempo de desarrollo menor
    \item Documentación clara 
    \item Aplicación del principio "Divide y Vencerás"
    \item Probable generación de comunidades alrededor de la herramienta
    \item Mayor eficiencia en tanto la librería se desarrolla 
\end{itemize}



