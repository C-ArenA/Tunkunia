\section{Alcances y Limitaciones}

De acuerdo a los objetivos planteados es menester delinear el campo de acción del proyecto.

\subsection{Alcances}

\begin{itemize}
    \item Se desarrollará un paquete de software en su primera versión con funcionalidades para facilitar la creación y seguimiento de trámites.
    \item Tunkunia será distribuido mediante al menos un repositorio remoto de versionado como \textit{GitHub}.
    \item El producto además será publicado en algún repositorio de paquetes de software coherente con la tecnología que use.
    \item Se usará una licencia de software libre que permita el uso de esta idea sin restricciones, pero mencionando al autor.
    \item El proyecto contará con la implementación de una documentación en línea para facilitar el uso de Tunkunia.
    \item Se demostrará el uso del paquete en una implementación de un \textit{fork} del sistema SIAI.
    \item Se usarán buenas prácticas de programación para lograr código legible y sobre el cual sea sencillo colaborar.
    \item Los lineamientos, fruto de la creación conceptual de este software serán creados con la ayuda de herramientas para desarrolladores como un CLI.
    \item Se recibirá al menos un \textit{pull request} en el repositorio para demostrar las bondades del software libre y se atenderá al menos un \textit{issue} reportado.
    \item Se adoptarán medidas de \textit{testing} para entregar un mejor código.

\end{itemize}

\subsection{Limitaciones}

\begin{itemize}
    \item El desarrollo no será realizado desde cero y de forma completa, sino que siguiendo los mismos lineamientos sobre los cuales se piensa este proyecto de software, se usarán herramientas de software libre como ladrillos de construcción para éste. Esto puede ser considerado un alcance de acuerdo a cómo se lo vea.
    \item La versión 1 será funcional, pero no se garantiza que esté libre de bugs, lo cual es natural en muchos desarrollos de software a pesar de la aplicación de las mejores prácticas y el testing.
    \item El sistema, si bien podría ser utilizado por un amplio abánico de aplicaciones, en su versión 1, que es la que atañe a este documento, se centrará unicamente en los trámites públicos más comunes y sin cubrir necesariamente las necesidades de cualquier trámite que pueda existir. Se deben tomar como marco los trámites realizados en el sistema SIAI, que es el caso de estudio sobre el cual se desarrollará Tunkunia.
    \item Si bien este proyecto puede ser aplicado en distintos entornos de manera conceptual, debe quedar claro que la implementación práctica de cualquier paquete de software suele estar acotada a ciertas tecnologías. En el caso presente esta tecnología será \textit{Laravel} por su presencia en el sistema SIAI y por su crecimiento reciente en popularidad.
\end{itemize}