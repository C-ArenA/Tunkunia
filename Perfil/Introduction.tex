\section{Introducción}

De acuerdo al reporte de las naciones unidas acerca de gobiernos electrónicos, se observa que el índice de desarrollo de gobiernos electrónicos (\textit{EGDI}), en la mayoría de los países, ha tenido un aumento significativo. Esto quiere decir que la adopción de las tecnologías de la información por los distintos gobiernos es una tendencia a nivel mundial.

En Bolivia, por ejemplo, el estado hace esfuerzos constantes por digitalizarse, como puede verse reflejado en el Artículo 71 de la Ley Nº 164 que declara prioridad nacional la promoción del uso de las tecnologías de información y comunicación para procurar el vivir bien de todas las bolivianas y bolivianos.

Esto no es una sorpresa, ya que los gobiernos electrónicos buscan modernizar y transparentar la gestión pública, brindando mayores beneficios a la ciudadanía con el uso de las TIC que, como se indica en el Decreto Supremo Nº 1793 de Bolivia, \textquote{se han convertido en medios esenciales para el desarrollo social, cultural, económico y político de los pueblos}.

Sin embargo, la digitalización de los gobiernos implica la implementación de software que pueda cubrir las necesidades muchas veces cambiantes de cada gobierno y su respectiva población. Software que estará presente en muchos espacios dentro del estado.

Dicho software ha de ser preferentemente de tipo FOSS (\textit{Free and Open Source Software}) y usar también herramientas de este mismo tipo, como se recomienda por ejemplo en el principio fundamental 4 del EIF (Marco Europeo de Interoperabilidad) o en el Artículo 77 de la Ley Nº 164 de Bolivia, para poder tener las ventajas de usabilidad y control soberano sobre las tecnologías usadas.

Una funcionalidad transversal en las aplicaciones del software para distintos organismos públicos es la de los procedimientos administrativos o, como se conocen comúnmente en países de habla hispana, los trámites.

Los trámites constituyen el conjunto de requisitos, pasos o acciones a través de los cuales los individuos o las empresas piden o entregan información a una entidad pública, con el fin de obtener un derecho o para cumplir con una obligación \cite{rosethFinTramiteEterno2018}. Trámites son también cada uno de los pasos, pero la palabra se usará en este documento bajo la definición anterior.

De esta manera, la propuesta que se realiza en este documento, trata sobre la creación de una herramienta reutilizable de software que ayude precisamente con los trámites, su implementación y seguimiento.

La solución desarrollada será un paquete de software libre dentro del ecosistema de \textit{Laravel} y buscará marcar una serie de recomendaciones para poder ser fácilmente replicada en distintas tecnologías.

El mismo paquete será aplicado en un fork del sistema SIAI del Ministerio de Desarrollo Productivo que fue desarrollado por la empresa 2IES y que cuenta con módulos de seguimiento de trámites. Se pretende dar un enfoque menos heterogéneo para cada trámite distinto con la ayuda de la herramienta a crear.

\subsection{Aclaraciones y Notas}
El resultado de este proyecto, al ser un paquete de software libre que será distribuido por internet, requiere identidad propia y será referenciado en adelante bajo el nombre de \say{Tunkunia}, o simplemente \say{el producto}.

Para evitar el uso inadecuado de términos técnicos, en esta redacción se prefiere la utilización de
términos en idioma inglés cuando el concepto es ambiguo en su traducción, no
existe una traducción oficial ampliamente utilizada o sencillamente para evitar
que este documento proporcione sus propias traducciones, como pasa en el caso de \say{librería de software}, proveniente del inglés \say{\textit{software library}}, cuya traducción correcta sería \say{biblioteca de software}, pero que algunos emplean para referirse a conceptos ligeramente distintos.